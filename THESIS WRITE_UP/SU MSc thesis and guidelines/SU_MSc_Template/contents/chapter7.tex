%\chapter{Conclusion and Future work}
%\label{cp1}
%
%This  research focused on comparing the  performance  of a L\'evy model in pricing vanilla and exotic options.  We first  calibrated the model parameters to the  S\&P 500 index option. We found that the CGMY model fits the S\&P 500 index  better than the VG, NIG and   Black-Scholes models. This was evidenced by the model prices with more degrees of freedom or model parameters fitting the observed market values better than one with a single model parameter (Volatility) as in the Black Schole  model. We  also showed that the vanilla prices obtained with the NIG, CGMY and VG models are similar, but they differ from prices obtained from the Black-Scholes model. The NIG and VG models give similar prices for the barrier call (Up-in and Up-out) which differed from the prices obtained using the CGMY and Black-Scholes models. Furthermore, the CGMY model fitted the market  better than the other models. The pure jump L\'evy model priced  the  European call, and the up-and-in and up-and-out calls options better than Black-Scholes model did.  It was noted that  the prices of the lookback fixed option obtained from each  model differed from one another. 
%
%We also calibrated CGMY, NIG, VG and BS models to the "CGMY-world" data (market prices obtained with the varying parameters of CGMY model), and we found that the new parameters of the CGMY model obtained through this calibration differed slightly to the varying parameters, even though the model "CGMY-world" data was computed with varying parameters. A further finding was that the NIG model fits the "CGMY-world" data better than the VG model, despite the fact that VG model is a particularly type of the CGMY model. We then priced the lookback and barrier options, and quantified model risk   using the new parameters calibrated to "CGMY-world" data. We noted that the lookback options are more sensitive to model risk when the options are out-the-money while barrier (up-and-in call) options are sensitive to model risk when options are in-the-money.
%
%%By hedging vanilla call options and also computing the  ratio, we observed that the pure jump model s  the vanilla call option better than the Black-Scholes models does. 	This means that the pure jump model may present less risk than the Black-Scholes model when hedging the vanilla call, especially  when the market prices expire out the money. 
%
%
% In general, it appears that the NIG, VG and CGMY models price the barrier, lookback fixed and vanilla call options better than the Black-Scholes model does.  Possible future research stemming from this thesis would be to compare the performance of the pure jump, and  Heston and Bate models with the Black-Scholes model when hedging the exotic option (especially the barrier option, lookback option and Asian option).
%
%%
%%In this research we are comparing the hedging performance for four L\'evy model in term of hedging ratio. We first note that the CGMY model, VG model and NIG model give a very good fit to S\&P 500 indexed than the Black-Scholes model. That is evident because the model prices with more degree freedom or model parameters  can fit better the observed market than one has one model parameter (Volatility) in the Black Schole model. This calibration procedure might not always lead to the same results if we try it or after a couple days or weeks. We observed that the pure jump L\'evy model give a  very similar European call prices and barrier call (Up-in and Up-out) than Black-Schole model. We found the price of lookback fixed option computed with these four model prices are very different to each other. That means the pure jump L\'evy model price better the European call option and Up-in and Up-out than Black-Scholes model. 
%%
%%This study was focused on comparing hedging performance for VG model, NIG model, CGMY model and Black-Scholes model as well. This comparison was based in term of  ratio. We considered that our market prices are obtained from CGMY model in order to avoid a lot of bias which can be found in observed market prices. We also fit all these four model prices  to the two different CGMY market price and computed hedging ratio for vanilla call option and lookback fixed call option. We found  that the Balck-Scholes model, VG model and NIG model  better the a European call option than the CGMY model. That means that when we use VG model, NIG model and Black-Scholes model to  an European call option the risk to lose the profit may be less than if we consider the CGMY model, especially when option prices are expired out-the-money. Now, by computing the hedging ratio for lookback fixed option. We obtained very interesting results. We found that the NIG model is only a L\'evy model amongst four L\'evy model prices which can  better the lookback fixed option. And the Black-Scholes model is not a right model that can be used in order to  the a lookback option.
%%
%%The another possible extension of this thesis may be in case  with more than one market price. This means we may consider four different market prices and three exotic option as Asian option, cliquet option and also barrier option. We compare the  performance for model as Heston model and Bate model where their close solution for their can be available. 
%
%%In this research we are compared the hedging performance for four L\'evy model in term of hedging ratio. We first note that the CGMY model, VG model and NIG model give a very good fit to S\&P 500 indexed than the Black-Scholes model. But it is evident because the model prices with more model parameters (VG Model, NIG model and CGMY model) can fit better the observed market than one has one model parameter (Volatility) in the Black Schole model. This calibration procedure might not necessary give a same or approximate result of model parameters if we try after a couple days or weeks.  We observe that the pure jump L\'evy model price give the similar prices of the vanilla call option and barrier call  (Up-in and Up-out) than Black-Schole model. While, we obtained the different price for lookback fixed option.  We may say that the Black-Scholes model is not an a good model to price exotic option. 
%%
%%Other part we assume that our market are obtained from CGMY model in order to avoid a lot of bias in market. While, they are still incomplete. We also fit all those three model to the two different CGMY market price and we compute hedging ratio of vanilla call option and lookback fixed call. We notice that the four L'evy model price give a very similar result of hedging ratio for a the vanilla call price. Thus, we can say that the four model present very similar risk when we price vanilla option even if they give a very different vanilla call price.  
%%
%%This study was focused on comparing  hedging performance of  VG model, NIG model, CGMY model and Black-Scholes model as well.  Indeed, we were which amongst four model may present less risk when we  vanilla call and lookback fixed call option. We used the hedging ratio which is composed by ratio  of the standard error (mean error between the portfolio value and vanilla or lookback fixed call option)  and the price given at expire time for vanilla call and lookback fixed call option. 
%%
%%The another possible extension of this thesis may be in case  with more than one market price. This means we may consider four different market prices and three exotic option as Asian option, cliquet option and also barrier option. Because many of  the research done or published did not focus in that part.