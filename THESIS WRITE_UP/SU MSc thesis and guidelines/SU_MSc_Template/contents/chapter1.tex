\chapter{Introduction}
\label{chp:Intro}
%%%%%%%%%%%%%%%%%%%%%%%%%%%%%%%%%%%%%%%%%%%%%%%%%%%%%%%%%%%%%%%%%%%%%%%
%\section{Introduction} 
Most sciences today use mathematical and computer simulation models to approximate the real-world processes under study \cite{Kennedy,Fojo,Vanni}. For example, models play a significant role in health policymaking by estimating the impact of interventions in situations where empirical studies may be time-consuming, costly and impractical \cite{Stout}. Developing a model calls for a trade-off between computational cost and accuracy; simple models require little computation time but can be a poor description of the real-world process, whereas complex models allow for a more accurate description of the process at the cost of increased computational cost. After model development, it is imperative to know how well the model represents reality. Model calibration, or fitting the model to data, increases the confidence that the model provides a realistic approximation to the real-world process \cite{Vanni,Stout}.\\
    
Calibration is the process of comparing model outputs with empirical data to identify the model parameter values that achieve a good fit to data \cite{Menzies,Vanni}. Calibration improves the credibility and validity of the subsequent predictions made and inferences drawn from the model \cite{Stout}. It is also commonly used in the case where model parameters are not observable or available, to estimate such input parameters \cite{Elske}.  The main components of calibration are summary statistics, parameter-search strategy, goodness-of-fit (GOF) measure and acceptance criteria. \\

Several methods have been used for model calibration and the number of studies that apply these calibration methods is proliferating in many research fields \cite{Vanni}. \cite{Stout} broke the model calibration process into seven stages which were later discussed in detail by \cite{Vanni}. \cite{Karnon} went through the seven stages of the calibration process using an early breast cancer model and produced a practical guidance on a more applicable calibration process. \cite{Vanni}, in their review article further examined different methods of calibration and reviewed some examples from health economic decision models. The model calibration methods applied in most studies are in two categories, optimisation methods and sampling methods  \cite{Menzies}. For the purpose of this study, we focus on the sampling methods.\\
 
Because there are many model calibration methods with little or no consensus on their performance, we perform a simulation study to compare the performance of model calibration methods using a simple stochastic Susceptible-Infected-Recovered (SIR) model. The methods to be compared are Rejection Approximate Bayesian Computation (Rejection ABC), Sequential Approximate Bayesian Computation (Sequential ABC) and Bayesian Maximum Likelihood estimation (BMLE).\\

Outline to be completed when thesis is fully written…….

