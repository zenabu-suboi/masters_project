\section*{Opsomming}
\addcontentsline{toc}{chapter}{Opsomming}
Dwelmmisbruik is 'n dreigende gevaar vir die gesondheid van beide dwelm gebruikers en nie-gebruikers. In die algemeen, word die misbruik van psigoaktiewe dwelms verbind met ho\"{e} risiko gedrag, mortaliteit en morbiditeit. Die dwelmgebruikskringloop behels onlosmaaklik vervlegde variante soos vervaardiging, handel en gebruik van beide wettige en onwettige verslawende middels. Die dinamika van dwelms behels aanvang, verslawing, rehabilitasie/behandeling en staking/herstel. In reaksie op die misbruik en verskaffing van monster dwelms, is beheer strategie\"{e} soos wetstoepassing en rehabilitasie verskerp, om die toegang tot dwelms te verminder, deur onderskeidelik te fokus op dwelmspilfigure en skadebeperking. 
Die belangrikste doel van hierdie verhandeling is om die faktore te modelleer wat die voorkoms van dwelmmisbruik be\"{\i}nvloed, die uitwerking van dwelmbase op die voorkoms van dwelmmisbruik, en die trefkrag van dwelmmisbruik op die voorkoms van MIV / VIGS. 
Ons formuleer wiskundige modelle gegrond op stelsels van outonome differensiaalvergelykings, wat die dinamika beskryf van die sub-bevolkinge wat in die dwelmgebruikskringloop betrokke is.   
Ons ondersoek die effekte van verbetering, rehabilitasie/behandeling en heraanvang op die voorkoms van dwelmmisbruik. 
Ons resultate dui dat, 
werwing tot rehabilitasie en verbetering in die teenwoordigheid van stakende tydelike verbruikers, die voorkoms van dwelmmisbruik verminder; heraanvang en verbetering sonder dat tydelike verbruikers staak, verhoog die voorkoms van dwelmmisbruik.
Ons raming van die invloed van dwelmbase en die uitwerking van wetstoepassing op dwelm-epidemies toon dat, die teenwoordigheid 
van dwelmbase belemmer grotendeels die pogings om dwelmmisbruik te verminder, aangesien hulle toegang tot dwelms verhoog. 
Nietemin, as die wetstoepassing verskerp word in reaksie op die dwelmbaasbevolking, word die voorkoms van dwelmmisbruik aansienlik verminder. 
Gegewe die gepaardgaande invloed van dwelms op ho\"{e} risiko gedrag as 'n kofaktor vir seksueel oordraagbare infeksies, beraam ons die invloed van dwelmmisbruik op die voorkoms van die Menslike Immunogebreksvirus (MIV). Ons resultate toon dat inligtingverspreiding rakende MIV en dwelmgebruik, MIV-voorkoms verlaag, terwyl daar 'n vinniger verspreiding van die epidemie en ho\"{e} voorkoms is, met verhoogde seksuele kontak.